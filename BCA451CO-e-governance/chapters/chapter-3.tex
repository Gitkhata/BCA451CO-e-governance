\chapter{ICT Infrastructure for e-Government}


\section{Network infrastructure}
Network infrastructure means the infrastructure that helps to connect computing devices within the office, other offices or connected to the world using internet. Network infrastructure includes networking devices like switch, router, networking cables, internet, intranet etc.

\section{Computing Infrastructure}
Computing infrastructure means availability of computers, laptops and other related devices which are needed for day to day computerized work in the office and providing various services.

\begin{itemize}
	\item While on one end, government needs large computing infrastructure to develop and deliver e-government services on continuous basis, infrastructure is also needed at the end of citizens to derive the benefits of these services.
	
	\item Again, like communication infrastructure, there is a high order of disparity in availability, affordability of computing devices in urban and rural areas, particularly in developing countries.
	
	\item Further, in rural areas, due to lack of basic infrastructure such as electricity, telephony, it may not be worthwhile for the people to have computers, even if they could afford it.
	
	\item To extend the reach of government services and address the wide range of citizens, governments all over the world are setting up common / shared / community infrastructure in the form of community information center, Internet kiosks etc.
	
	\item Government should also consider, making their services accessible from various other media/devices such as basic telephones, mobiles, cable TV network, PDAs and many other hands held devices.
	
\end{itemize}
\section{Data Centers}
Data centers is a place where many dedicated computers, servers and storage are available for mass storage of data. All public or private offices can backup their important data in secure way in least cost and prevent any loss of their data from disaster or failure of their system. In Nepal there are two data centers GIDC and DOIT. 

\begin{itemize}
	\item In the era of e-governance, government is expected to deliver its services to the citizens on $ 24\times7 $ bases. To achieve this, the government has to set up a sound and stable infrastructure operational round the clock.
	
	\item Internet Data Center is a facility which provides extremely reliable and secure infrastructure for running Internet operations on a $ 24\times7 $ basis. It shall not at all be cost effective if each department starts setting up its own data center as running a high class Internet Data Center needs a lot of recurring resources.
	
	\item It is, therefore, suggested that the government may set up a high grade Data Center at a National level to be used by all entities of the government.
	
	\item All departments should, in turn, establish high speed connectivity with the data center so that they can manage their applications from their own premises in a secured manner.
	
	\item In cases where the country is large and the government feels that one Internet Data Center may not suffice, it could decide to set up multiple Data Centers.
	
	\item However, the number of data centers should be optimized to the extent possible primarily due to the high recurring operative costs as well as scarcity of skilled resources.
		
	\item As the pace of e-government picks up nationwide, besides delivery of services, Government may also have to set up data centers to share the large scale/special purpose resources for development of the systems.
\end{itemize}


\section{e-Government Architecture}
There is no commonly agreed definition of e-Government architecture. The result is how the different countries states it. E-government Architecture generally consists of three components: 
\begin{multicols}{2}
	\begin{enumerate}
		\item Service Architecture
		\item Process Architecture and 
		\item Data Architecture
	\end{enumerate}
\end{multicols}


\subsection{Service Architecture}
Describes a lot of services offered by the Government, processes to be followed for each service, Concerned Department(s), relation/dependence on other services etc. Services could be like Vehicle Registration, Passport Issuance, Caste Certificate, Payment of Tax, etc.

 \subsection{Process Architecture}
 
 \begin{enumerate}[label=(\roman*)]
 	\item Lists the various processes to be followed for rendering different services, independent of their association with one or more services. 
 	\item These processes are then further grouped in various categories and detailed rules/procedures are defined for executing each of the processes. 
 	\item This brings a lot of standardization across services and promotes interoperability as well as reuse of process components. 
 	\item Processes could be Content Management, Citizen Registration, Personalization, Online Form Submission, Electronic Payment etc.
	 \end{enumerate}

\subsection{Data Architecture}

 \begin{enumerate}[label=(\roman*)]
 	\item Deals with the data associated with various Government Services, as described in service architecture. 
 	\item In Data Architecture, we enlist all the data elements needed/associated with above service and then define meta-data about each data element.
 	\item This meta-data information includes the standard Nomenclature for each data elements, their type, size, format, default value, valid value range, owner etc. 
 	\item Use of such a standard definition by all government applications shall facilitate interoperability among various applications as well their integration which shall go long way in delivery of integrated / one stop services to the citizens and businesses.
 \end{enumerate}


\section{Interoperability Framework}
\begin{itemize}
	\item The Interoperability Framework aims to define the set of specifications to facilitate Government systems to communicate and interoperate with other systems, both within Government and external to it, efficiently and effectively.
	\item By bringing together the relevant specifications under an overall framework, ICT management and software developers have a single point of reference whenever a need arises to locate the required interoperability specifications that should be followed for a specific project. 
	\item By adopting these interoperability specifications, system designers can ensure interoperability between systems while at the same time have the flexibility to select different hardware, systems and application software to implement solutions.
	\item In order to attain this objective, the Government needs to be perceived as a single entity, with seamless flow of information across individual ministries and departments as necessary.
	\item Framing of policies and specifications for Interoperability Framework should be followed up with provision of support, guidance on best practices, toolkits and agreed schema. 
	\item The entire strategy to implement good e-government should be viewed in long-term perspective and hence must be supported by vigorous processes. 
	\item The development of Interoperability Framework must therefore be reviewed and updated on a continuous basis.
\end{itemize}

\subsection*{e-Government Interoperability Framework (e-GIF)}
e-GIF is a set of guidelines and technical specifications, designed to promote interoperability between various e-government systems, though developed independently by various agencies. In the words of Mr. Douglas Alexander, in his foreword to the e-GIF Framework (Version 5.0),

\begin{quotation}
	\noindent \say{in terms of e-service delivery, compliance with the (e-GIF) Framework is essential for the public good... the Framework aligns government with the rest of the industry and serves as a basis for reducing the costs and risks associated with carrying out major IT projects}
\end{quotation}

The key policy decisions that have shaped the e-GIF are as follows:

\begin{itemize}
	\item Align with the \textbf{Internet technologies} and specifications, for all public sector information systems.
	\item Adopt \textbf{XML} as the primary standard for data integration and presentation tools,
	\item Adopt the \textbf{browser} as the key interface.
	\item Adopt \textbf{metadata} to government information resources.
	\item \textbf{Mandate} e-GIF throughout the public sector.
\end{itemize}

The e-GIF specifications are driven by integrability, market-support, scalability and openness of the technologies prescribed. The scope of e-GIF specifications has been limited to four key ares of technology, viz.

\begin{itemize}
	\item Interconnectivity,
	\item data interpretation,
	\item e-services access  and 
	\item content management
\end{itemize}

\newpage\thispagestyle{empty}
