\chapter{Public-Private Partnership for e-Government}
Public Private Partnership (PPP) is a different method of procuring public services and infrastructure by  combining the best of the public and private sectors with an emphasis on value for money and delivering quality public services.

The concept of PPP has been brought into operation in the construction and operation of public infrastructure projects like bridges, airports, highways, hospitals, etc.

PPP is a reform that is a `generation next' to privatization. Privatization is the process of involving the private sector in the ownership and management of of ongoing and existing projects and business of the public sector. In PPP, the private sector partner is induced into a project right from the stage of initiation to completion and management.

\section*{Why PPP for e-Government?}

\subsection*{Combining Accountability With Efficiency}
\begin{itemize}
	\item PPP for e-government would combine the accountability and domain expertise of the public sector with the efficiency, cost-effectiveness and customer-centric approach of the private sector.
\end{itemize}

\subsection*{Complexity and Size of e-government}
Since many agencies are managed by government, its strucute is huge and complex but government does not have sufficient resource to manage such complexities, private sector could raise \textit{unlimited} resources.

\subsection*{Pace of Implementation}
Government cannot plan for implementing projects one after the another because it would be almost impossible to maintain all the projects smoothly. 

In order to maintain a high pace in implementing e-government, government should join hands with the private sector.

\section{G2B Project}
\begin{itemize}
	\item e-procurement
	\item G2B portal
\end{itemize}

\section{G2C Project}
\begin{itemize}
	\item Citizen service portals
	\item Integrated service centers
	\item Agency service centers 
	\item Networks of kiosks
\end{itemize}

\section{PPP Forms}
PPP can be of different forms, depending on the shares of government and the private sector in the investment, control as also on the strategic nature and commercial viability of the project/initiative. Different models of PPP are described in the following section.

\subsection[JV Model]{Joint Venture (JV) Model}
In this model, an SPV (Special Purpose Vehicle) is formed to undertake the e-government project and/or to provide e-services. The joint venture can be led by the government or by the private sector depending upon the strategic nature and sensitivity of the domain.

A JV model is preferred option for projects involving
\begin{enumerate}[label=(\alph*)]
	\item delivery of services, which are basic and permanent in nature e.\ g.\ a country portal.
	\item setting up of infrastructure with steady returns envisaged in long term e.\ g.\ a State Data Center.
	\item handling of sensitive data and information relating to citizen, businesses and government and
	\item close coordination with and cooperation from  a host of government agencies.
\end{enumerate}

In Joint Venture the government share varies from 51\% to 11\% which can be in cash but also can be in the form of tangible assets like land, building, equipment or in the form of intangible assets like right to access government information and databases for providing e-services.

In Nepal, some hydro projects are under construction on JV model between government and public.

\subsection[BOO Model]{Build-Own-Operate (BOO) Model}
\begin{itemize}
	\item In this model, the selected partner designs, develops and implements the projects, most often, entirely at its cost and operates the system for a pre-specified period called \textit{concession} period.
	\item The revenue model of the project is either based on transaction charges (paid by the citizen or the government) or EQI/EMI(Equated Quarterly Installment/Equated Monthly Installment) paid by the government to the operator/service provider.
	\item The BOO model is suitable for projects that involve setting up of physical infrastructures such as service center(s) for delivering services to the citizens. 
\end{itemize}

Example are projects related to:
\begin{multicols}{2}
	\begin{itemize}
	\item driving licenses
	\item vehicle registration
	\item provision for integrated services
	\end{itemize}
\end{multicols}

The important aspects in drafting Request For Proposal (RFP) for BOO Project are:
\begin{enumerate}[label=(\alph*)]
	\item to determine period of the arrangement during which partner is authorized to deliver the services, and
	\item The bid parameter dealing with transaction charges and/or EQI/EMI to be quoted by the competitive bidders.
\end{enumerate}

The BOO model is usually adopted in e-Government projects that deploy time-tested technologies and have a fairly reliable revenue mode.

\subsection[BOOT Model]{Build-Own-Operate-and-Transfer (BOOT) Model}
This is almost identical to BOO except that the government exercises ownership of the assets created by the partner at the end of the project. 

This model is adopted where the technology is time tested and the ICT assets are expected to outlast the concession period.

\subsection[ASP Model]{Application Service Provider (ASP) Model}
In this model the government contracts to avail\footnote{Take or use} the services of the partner for delivery of services as per mutually agreed service levels and commercial terms. The revenue model is typically transaction based. The ASP model is suitable to e-government initiatives that involve:

\begin{enumerate}[label=(\alph*)]
	\item a requirement to launch the services in a short time frame.
	\item the technology is not complex and is widely accepted and practiced in the private sector, and
	\item the nature of information is not so sensitive or critical to governance.
\end{enumerate}

Examples of ASP model are:
\begin{enumerate}[label=(\roman*)]
	\item design and hosting of websites that provide fairly static information to the citizens.
	\item provision of simple services like downloading/filing of forms, and
	\item provision of MIS services in the G2G arena to the government agencies.
\end{enumerate}

Most often, the ASP model is useful to leverage the existing ICT infrastructure and management skills already established by service providers. This creates a win-win situation by enabling the optimum utilization of the ICT infrastructure already setup in the private sector and thereby reducing the transaction cost to the government/citizen. The ASP model also saves the government agencies of the hassles of designing complex technology and partnership models.


\section{Issues in PPP for e-Government}
Though there are many advantages of PPP, if proper negotiation is done. But both have their own interest to earn more benefits which may threaten PPP relationship. Some issues are:

\subsection{Lack of Congruence in Objectives}
The degree to which the public and private sector partners align themselves along sharing the investment and control. Both must commit to developing an understanding of each others objectives but failing in such understanding and only focus on own interest creates lack of congruence in objectives and may fail the relationship.

\subsection{Risk and Control}
In every business there is risk and control mechanism. Most often, governments attempt to transfer risk to the partner without passing on the related control quoting ‘public interest’ as the reason.

\subsection{Clash of Cultures}
The organizational culture of private and public sector differ widely in all parts of the world which is bound to result in conflicting situation. The private partners tend to look at the government employees as bureaucrats with antiquated ideas that have outlived their time. 

\subsection{Monopoly}
In some cases, only one partner is suitable in areas such as e-procurement, country or state portal, data center, gateway and the like. This is likely to result in a situation of monopoly- the monopoly of the state being replaced with the monopoly of the private partner and more importantly, \textit{monopoly of a particular technology}.

The following methodology is recommended to mitigate its impact.

\subsubsection*{Operational Monopoly}
The \textit{operational monopoly} can be handled by defining the commercial features of the contract unambiguously while notifying the project to an open bid. The following factors are to be considered:
\begin{enumerate}
	\item Projected customer base and transaction volume
	\item Length of the concession period
	\item Fee structure of the existing services
	\item Price elasticity of the new services
	\item Capacity for growth.
\end{enumerate}

\subsubsection*{Technology Monopoly}
The \textit{technology monopoly} can be mitigated by prescribing open standards in conformity with the technology architecture approved by the government and ensuring that there is scope for developing interfaces with other systems that may be developed concurrently or in the future.

\section{Citizen-Centric Approach to e-Government}
Citizen-centric eGovernment services are designed to deliver increasingly costeffective, personalized and relevant services to citizens, but also serve to enhance
the democratic relationship, and build better democratic dialogue, between citizens
and their government, which then enhances the practice of citizenship within
society.

\begin{enumerate}
	\item It is necessary to look at e-government from the citizen or customer’s point of view and design the front-end and the back-ends to the extent required to fulfill the requirement of citizen/customer, i.\ e.\ e-government initiatives should not be system driven or supply-driven but should be demand-driven.
	
	\item The e-Government projects can be classified as \textit{core} and \textit{non-core}. Core projects are those, that can be used by all departments across the state and with significant impact on key stakeholders like citizen, businesses and employees.
	
	\item The e-Government projects  can also be categorized as \textit{commercial} and \textit{non-commercial}. Commercial projects are those that permit a viable public-private partnership model to be implemented with the least outgo from the public exchequer\footnote{The funds of a government} for implementation.
\end{enumerate}

\newpage\thispagestyle{empty}