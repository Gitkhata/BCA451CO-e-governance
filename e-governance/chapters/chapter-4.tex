\chapter{e-Government Readiness}
e-government takes root and grows when a country, state or agency is \textit{e-ready}.

\subsection*{e-Readiness}
According to Harvard Business School,
\begin{quotation}
	\noindent an e-Ready society is one that has the necessary physical infrastructure (high bandwidth, reliability and affordable prices). It should also have an integrated, current ICT's throughout business communities (e-commerce, local ICT sector), and government (e-government). Other important aspects are strong telecommunications competition, independent regulation with a commitment to universal access, and no limits on trade or foreign investment.
\end{quotation}


In short, e-readiness measures a nation's capacity to participate in the digital economy. While e-readiness is a larger concept that measures how a nation comprising citizens, businesses and government takes advantage of the digital revolution, `e-government readiness' relates to how the process involving the government are transformed using the tools of ICT. In other words, e-readiness touches upon the state of all interface - G2G, G2B, B2B, B2C and C2C, while e-government readiness is concerned with only the first three interfaces.


\section{e-Readiness framework}


Following list shows component, sub-component and indicators of e-readiness.

\begin{enumerate}
	\item \textbf{Policy}
	      \begin{enumerate}
		      \item ICT Policy
		            \begin{itemize}
			            \item Communications Policies
			            \item Policy on ISP
			            \item Incentives to ICT Industry
			            \item Recognition of Quality
			            \item Facilitation of Growth \& Promotion of Exports
		            \end{itemize}

		      \item E-Government Policy
		            \begin{itemize}
			            \item E-Government Vision
			            \item Prioritization of Services
			            \item PPP Policy
			            \item Policy on ESD (Electronic Service Delivery)
		            \end{itemize}

		      \item Architecture \& Standards
		            \begin{itemize}
			            \item Functional Architecture
			            \item Technical Architecture
			            \item Technical Standards
		            \end{itemize}

		      \item Security Framework
		            \begin{itemize}
			            \item Security Policy
			            \item Privacy
		            \end{itemize}

		      \item Regular Framework
		            \begin{itemize}
			            \item Cyberlaw
			            \item IPR Protection
		            \end{itemize}
	      \end{enumerate}

	\item \textbf{Infrastructure}
	      \begin{enumerate}
		      \item Networks
		            \begin{itemize}
			            \item National Backbone(s)
			            \item Distribution Networks
			            \item LANs \& WANs
			            \item Satellite \& Wireless Networks
		            \end{itemize}

		      \item Access
		            \begin{itemize}
			            \item PC Penetration
			            \item Internet Penetration
			            \item Last Mile Connectivity
		            \end{itemize}

		      \item ICT Hardware
		            \begin{itemize}
			            \item Data Center
			            \item e-Government Gateway
			            \item Payment Gateway
			            \item Public Key Infrastructure
		            \end{itemize}
	      \end{enumerate}

	\item \textbf{Resources}
	      \begin{enumerate}
		      \item Political Resources
		            \begin{itemize}
			            \item Leadership \& Vision
			            \item Continuity of Support to ICT Sector
		            \end{itemize}

		      \item Human Resources
		            \begin{itemize}
			            \item IT Education \& Training Institutions
			            \item Expenditure on R\&D in ICT
		            \end{itemize}

		      \item Employee Resources
		            \begin{itemize}
			            \item Champions of ICT
			            \item Chief Information Offices
			            \item Access to PC \& Internet at Office
		            \end{itemize}
	      \end{enumerate}

	\item \textbf{Usages}
	      \begin{enumerate}
		      \item Usage by Citizen
		            \begin{itemize}
			            \item e-Mail \& Internet Usages
			            \item e-Literacy

		            \end{itemize}

		      \item Usage by Businesses
		            \begin{itemize}
			            \item e-Commerce
			            \item e-CRM; e-SCM
			            \item e-Procurement in B2B \& G2B Areas
		            \end{itemize}

		      \item Employee Resources
		            \begin{itemize}
			            \item No.\ of Websites/Portals
			            \item No.\ of e-Services; e-Transactions
			            \item No.\ of e-Government Projects
			            \item Extent of G2G Usage
		            \end{itemize}
	      \end{enumerate}
\end{enumerate}

The e-readiness framework consists of assessing readiness along four fronts:
\begin{itemize}
	\item policy,
	\item infrastructure,
	\item resources and
	\item usages.
\end{itemize}

Each of the four components consists of 3-5 sub-components that enable a deeper understanding of the state of each of the major components. A set of 43 indicators is suggested as a drilled down of the sub-components to enable quantitative and qualitative assessment of e-readiness.

It is possible to develop a methodology to asses the e-readiness of a country or a state, through a structured questionnaire, administered to a representative sample population of the citizens, companies and government agencies and supplementing the same with the macroeconomic data available with regulatory bodies, research institutions and industry associations. The following questionnaire is suggested as a starting point. It can be improved upon and customized to suit varying circumstances.

\subsection{Policy}
\subsubsection*{ICT Policy}
\begin{itemize}
	\item Does the country (State) have an ICT policy? How old is it? How contemporary is it?
	\item Does the country (State) have a telecommunications policy? Does it promote competition?
	\item Does the country have a policy on ISPs (Internet Service Providers)?
	\item Does the State provide sizeable incentives to the ICT sector in the form of tax concessions, allocation of state lands at a concession?
	\item Does the state promote exports of ICT products and services?
	\item Are there awards instituted for excellence in ICT sector?
\end{itemize}

\subsubsection*{e-Government Policy}
\begin{itemize}
	\item Is there a document that specifies the state's e-government vision and strategy?
	\item Is there clarity on the priorities in implementation of e-government? Is there a 5- or 10-year perspective plan, broken down into annual action plans with clear quantitative and qualitative targets?
	\item Has the state laid down a transparent policy on Public-Private Partnership for e-government? How many PPP initiatives are ongoing/completed?
	\item Is there a policy on ESD that creates an open framework for development of multiple channels?
\end{itemize}


\subsubsection*{Architecture and Standards}
\begin{itemize}
	\item Has the government published a document that sets out the functional architecture or business process architecture?
	\item Is there a document that prescribes standards in all the technology areas like application development, databases, middlewares, networks, storage, etc.\ ?
	\item Has the government published a technology architecture that is based on open standards and permits development of IT systems in an interoperable manner and a seamless integration with national and global systems?
\end{itemize}



\subsubsection*{Regulation}
\begin{itemize}
	\item Is there an overarching cyberlaw at the national level that confers legal status to electronic transactions and documents?
	\item Is there a law on regulation of digital signatures and encryption?
	\item Is there a law to protect Intellectual Property?
	\item Is there a law on privacy that protects the information of citizens and businesses captured by the government and private agencies, against unauthorized use? 
	\item Is there a statutory regulator for the telecom sector that promotes competition?
	\item Is there an effective legal machinery to tackle the problem of piracy of ICT products?
\end{itemize}

\subsection{ICT Infrastructure}
\subsubsection*{Networks}
\begin{itemize}
	\item Are there at least two major national networks that connect all the major cities? 
	\item Are there two or three distribution networks to connect all towns and all villages? 

	\item Do the federal and state governments and their agencies have WANs of their own? 

	\item Do the public offices and enterprises have LANs that use State-of-the-art switches and routers? 
	\item Is there effective usage of satellite and wireless networks in government and business?
\end{itemize}

\subsubsection*{Access}
\begin{itemize}
	\item What is the PC penetration in terms of a PC per 1000 population? What percent of households have PCs?

	\item What is the Internet penetration in terms of Internet accounts per 100 of population?

	\item What is the technology adopted to connect the last mile? Dial up? Optical Fiber Cable? Wireless?
\end{itemize}

\subsubsection*{ICT hardware}
\begin{itemize}
	\item How many data centres are established in the country?

	\item Does the country have an e-government gateway?

	\item Is there an e-payment getaway?

	\item Does the country have a PKI?
\end{itemize}

\subsection{Resources}

\subsubsection*{Political Resources}
\begin{itemize}
	\item Is there a document at national level that describes the ICT vision of the country?
	\item Is there a political consensus on the promotion of ICT in the country? 
	\item Are there champions of ICT and e-government at the national and state levels among the political executives?
	\item In the last 10 years, how often has political support been given to the ICT sector?
\end{itemize}

\subsubsection*{Human Resources}
\begin{itemize}
	\item Is the country or state self-sufficient in IT graduates? 

	\item How many IT training institutes operate at the national level, outside the formal education system?

	\item What is the percentage of turnover spent on R\&D in the ICT sector?

	\item How many institutions of excellence that have national and international reputation exist in the IT sector?
\end{itemize}

\subsubsection*{Employee Resources}
\begin{itemize}
	\item What percentage of enterprises (other than SMEs) have qualified Chief Information Officers?
	\item What is the percentage of government enterprises- federal and state-having CIOs?
	\item What is the percentage of employees having access to a PC and Internet at office — in the public and private sectors?
\end{itemize}

\subsubsection*{ICT Resources of Private Sector}
\begin{itemize}
	\item How many ICT companies are active in the country/state?
	\item How many of them partner the government?
\end{itemize}

\subsubsection*{Financial Resources}
\begin{itemize}
	\item What is the total ICT budget of the Federal, State and local governments? 
	\item What is the annual IT expenditure of the private sector?
\end{itemize}


\subsection{Usage}
\subsubsection*{Usage by Citizens}
\begin{itemize}
	\item What is the rate of e-literacy among the citizens?

	\item What is the extent of e-mail usage and Internet browsing among citizens?

	\item What percentage of citizens use e-service over the Internet in preference to over-the-counter?

	\item What is the share of e-buying in the total consumer spend?
\end{itemize}

\subsubsection*{Usage by Business}
\begin{itemize}
	\item What is the share of e-commerce in the overall business?

	\item What percent of major industries and business have adopted eCRM, eSCM and e-procurement?

	\item What is the level of trust in the net among business people?
\end{itemize}

\subsubsection*{Usage by Government}
\begin{itemize}
	\item What percent of G2C and G2B services are offered electronically?
	\item What percent of G2G, G2B transactions occur electronically?
	\item How many enterprise-wide e-government projects are operational?
	\item What percent of government agencies have websites/portals that are regularly updated and used by the citizens?
	\item What percent of government employees use PC and Internet for official work?
\end{itemize}

It is a long and elaborate questionnaire. The answers to most of these questions can be qualitative to begin with. Where quantitative responses are required, a sample survey would be the best. It is advisable to adopt a system of weightages to assess the overall e-readiness of a country/state or enterprise.

\section{Steps to e-Government Readiness}
10- Step process to e-government readiness that can act as a guide for improving the score of e-government readiness. It is not necessary to follow the 10 steps sequentially. Some of them can be implemented in parallel. Each step may be broken down into a set of tasks and pursued for effective results. In fact, some steps and components, such as design of architecture, the CIO program, setting up of a state data center and gateway, are themselves very large initiatives.

\begin{steps}
	\item Articulate the e-government vision and strategy. Prepare a five-year perspective plan.

	\item Review the Telecommunication policy, to promote an open, competitive environment for creation of national and sub-national networks.

	\item Prepare a list of G2C and G2B services that citizens and businesses need to be provided electronically.

	Prioritize the services.

	Announce a policy on electronically services delivery.

	\item {\label{stp:four}}Design Functional and Technology Architectures that are aimed at delivering the e-services.

	Prescribe standards for security.

	\item Initiate statewide e-government projects adopting the pilot approach. Ensure these are part of the ‘big picture’ developed in {\ref{stp:four}}.

	\item Design and implement an appropriate CIO program.

	Implement change management programs across all major government agencies.

	\item Ensure that all government agencies earmark\footnote{designate (funds or resources) for a particular purpose} 2-5\% of their budget to e-government.

	Announce a PPP policy for e-government and take up a few projects adopting the PPP
	Model.

	\item Establish a government–wide WAN for data, voice and video for G2G applications, adopting a PPP model.

	\item Enact a cyber law that gives a legal validity to all electronic transactions and records and permits use of digital signatures for authenticating messages and documents.

	Publish polices on security and privacy for e-government.

	\item Establish data centers for e- government using the PPP model.

	Design and establish an e-government gateway at the State Data Center.
\end{steps}

\section{Issues in e-Government Readiness}
Getting a country into a stage of e-readiness requires a mulltipronged\footnote{having several distinct aspects or elements} effort. While it is possible to adopt a structured approach, it is fraught\footnote{(of a situation or course of action) filled with or likely to result in (something undesirable)} with several problems. It is necessary to look at three issues which are crosscutting in nature:
\begin{multicols}{2}
	\begin{itemize}
		\item people readiness,
		\item reform readiness, and 
		\item readiness for sustainability.
	\end{itemize}
\end{multicols}



\subsection{People readiness}
We can program processes. We cannot program people. There lies the problem — in getting people ready for e-government. People readiness has four stages of evolution.
\begin{multicols}{2}
	\begin{enumerate}
	\item Readiness to \textit{think}
	\item Readiness to \textit{learn}
	\item Readiness to \textit{act}
	\item Readiness to \textit{transform}.
\end{enumerate}
\end{multicols}


\subsubsection{Readiness to Think}
Readiness to think of e-government is to do with the change of mindset and is by far the most difficult one to achieve.

\subsubsection{Readiness to Learn}
Readiness to learn is easier to come by. It can be ushered\footnote{show or guide (someone) somewhere.} in through a set of attractive training programs coupled with visits to successful e-government projects and interaction with people running the project and those benefiting from it.

\subsubsection{Readiness to Act}
Readiness to act is a hands-on exercise. It is believed that giving a person a PC and exposing to the Net is a good way to initiate him or her into the e-world and getting people hooked on to `Act'.

\subsubsection{Readiness to Transform}
Readiness to Transform is the final stage where people in the organization start acting as teams, willing to spare an extra hour to improvise, improve, innovate and transform the workplace and the service center.

\subsection{Reform Readiness}
e-government efforts end up as `old wine in new bottle' unless these are accompanied by an urge to transform the way government functions and treats its customers. This is possible through an extensive exercise to reform the processes and the legal provisions underlying them. Reform is triggered by the need to introduce new services and to provide the existing services in a new way to the citizens, in a manner that is convenient and cost effective from the citizen's viewpoint.


\subsection{Backend Readiness vs. Front-end Readiness}
One of the classic conflicts that arises in the course of a serious implementation of e-government is the one between `backend readiness' and `front-end readiness'. By `backend readiness' we mean the following tasks:

\subsubsection*{Developing Backend Systems}
\begin{itemize}
	\item Design of e-services
	\item Business process reform
	\item Development of application software
	\item Pilot and rollout
\end{itemize}


\subsubsection*{Establishment of Iinfrastructure}
\begin{itemize}
	\item Establishment of a data center
	\item Setting up of hardware at all agency locations
	\item Networking of all backend systems
\end{itemize}

\subsubsection*{Readying the people}
\begin{itemize}
	\item Creation of a cadre of CIOs
	\item Training
	\item Change management
\end{itemize}

\subsubsection*{Front-end} 
Frontend readiness means the following:
\begin{itemize}
	\item Creation of a delivery channel policy
	\item Establishing service centers/kiosks
	\item Creation of websites and portals
\end{itemize}

Emphasis on the front end readiness produce quick results and impact in short run which is necessary in generating the excitement required to attract people- employees and citizens. Launching of information websites, online statistical systems, etc.\ are typical examples of the eagerness to bring in quick visibility through front-end cosmetics. However, excessive stress on the front-end without backend readiness is dangerous. This leads to disillusionment.


\newpage\thispagestyle{empty}
