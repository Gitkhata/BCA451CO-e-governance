\documentclass[a4paper, twoside, 12pt, noanswers]{exam}


% \documentclass[a4paper, 12pt, answers]{exam}
\usepackage{fontspec}

\usepackage{polyglossia}
\usepackage{dirtytalk}
\usepackage{enumitem}
\usepackage{fancybox}
\usepackage{framed}

\setdefaultlanguage{english}


\setotherlanguage{hindi}
\newfontfamily\devanagarifont[Script=Devanagari]{Shobhika} % Shobhika, Mukta, Kalimati, Lohit Devanagari, Lohit Nepali

\defaultfontfeatures{Ligatures=TeX}


\newcommand{\textnp}{\texthindi}
\newenvironment{nepali}{\begin{hindi}}{\end{hindi}}



\usepackage[onehalfspacing]{setspace}
\usepackage{graphicx}
\usepackage{multicol,adjustbox}

% \pointsinrightmargin %points in right margin
% \bracketedpoints

\nopointsinmargin
\pointformat{}

\unframedsolutions


\CorrectChoiceEmphasis{\bfseries\boldmath}
\renewcommand{\thepartno}{\alph{partno}}
\renewcommand{\thesubpart}{(\Roman{subpart})}
\renewcommand{\subpartlabel}{\thesubpart} 
\renewcommand{\choicelabel}{(\alph{choice})}
\thispagestyle{empty}

% \title{\vspace{-1.7cm} 
% 	\hrule
% 	\vspace*{0.2cm}

% 	{\bfseries\LARGE PURBANCHAL UNIVERSITY \\2018}\\[2mm]
% 	{\large 4 Years Bachelor of Computer Application (BCA/Eighth Semester/Final)}\\[2mm]	  
% 	{\large Time: 3.00 hrs.  \hfill Full Marks: 60 /Pass Marks: 24}\\[2mm]
% 	{\large \bfseries {BCA452CO, Multimedia Application \hfill {}}}\\[2mm]
% 	\hrule
% }
% \date{}

\cfoot[]{}
\clearpage

\begin{document}
% \maketitle\vspace{-2.5cm}

\begin{framed}
	\raggedright{\bfseries\Large\centering PURBANCHAL UNIVERSITY \par {2018/ \textnp{२०७५}}\par}
	{ 4 Years Bachelor of Computer Application (BCA/Eighth Semester/Final)\par}
	{ Time: 3.00 hrs.  \hfill Full Marks: 80 / Pass Marks: 32\par}
	{\bfseries {BCA451CO, e-Governance \hfill}\par}
\end{framed}
% \begin{framed}
% 	\raggedright{\bfseries\Large\centering {\textnp{पूर्वाञ्चल विश्वविद्यालय}}\par {\textnp{२०१८} }\par}
% 	{ \textnp{४ वर्षे ब्याचलर अफ कम्प्युटर एप्लिकेसन (वि.सि.ए) अ‍ाठाैँ सेमेस्टर}\par}
% 	% 4 Years Bachelor of Computer Application (BCA/Eighth Semester/Final)\par
% 	{ \textnp{समयः ३ घण्टा}.  \hfill \textnp{पूर्णा‌ङ्क : ६० / उत्तीर्णाङ्क : २४}\par}
% 	{\bfseries {BCA452CO, Multimedia Application \hfill}\par}
% \end{framed}
{\noindent \it{Candidates are required to give their own answers in their own words as far as practicable.  }\par}
{\noindent \it{Figure in the margin indicate full marks.}\par}

%%%%%%%%%%%%%%%%%%%%%%%%%%%%%%%%%%%%%%%%%%%%%%%%%%%%%%%%%%%%%%%%%%%%%%%
{\fullwidth{\centering \bfseries \underline{Group A}}}
{\hspace*{-0.5cm} \bfseries Answer TWO questions.} \hfill {\( \mathbf{2\times 12 =24}\)}
%%%%%%%%%%%%%%%%%%%%%%%%%%%%%%%%%%%%%%%%%%%%%%%%%%%%%%%%%%%%%%%%%%%%%%%

\begin{questions}

	\question[3+5+4] Compare between e-Government with e-Commerce. Explain development stages of e-government and barriers of e-
	Government implementation with example.
	
	\question[4+8] Explain network infrastructure. Why e-Government architecture and interoperability frameworks are needed for e-Government infrastructure development? Explain with an example.
	
	\question[8+4] Discuss e-government security architecture. Also explain the challenges to implement e-government system.

	

%%%%%%%%%%%%%%%%%%%%%%%%%%%%%%%%%%%%%%%%%%%%%%%%%%%%%%%%%%%%%%%%%%%%%%%
{\fullwidth{\centering \bfseries \underline{Group B}}\par}
{\hspace*{-0.5cm}\noindent\bfseries Answer SEVEN questions.} \hfill {\( \mathbf{7 \times 8 = 56}\)}
%%%%%%%%%%%%%%%%%%%%%%%%%%%%%%%%%%%%%%%%%%%%%%%%%%%%%%%%%%%%%%%%%%%%%%%

\question[4+4] Describe importance of e-Readiness with e-Readiness framework. Explain issues in e-government readiness. 

\question[2+6] What is PPP? Describe citizen-centric approach to e-Government adoption. 

\question[4+4] Explain management approaches of e-Government system. Describe emerging management issues for e-Government system. 

\question[8] Explain the e-Government risk assessment and mitigation with an example.

\question[4+4] What are the major task involved in implementing e-Government. Describe analysis of current reality.

\question[8] Show comparative analysis of e-Government development in South Korea and China.

\question[3+5] What is e-Government system life cycle? Explain security management model in detail.

\question[4+4] Write short notes on any TWO:
\begin{parts}
	\part e-Government master plan
	\part Managing public data in e-Government.
	\part Impact of e-Suvida and Cyber Laws in Nepal
\end{parts}

\end{questions}
\newpage


%%%%%%Purbanchal University-2019%%%%%%
% 									 %															
%				2019				 %
%									 %
%%%%%%Purbanchal University-2019%%%%%%

\begin{framed}
	\raggedright{\bfseries\Large\centering PURBANCHAL UNIVERSITY \par {2019/ \textnp{२०७६}}\par}
	{ 4 Years Bachelor of Computer Application (BCA/Eighth Semester/Final)\par}
	{ Time: 3.00 hrs.  \hfill Full Marks: 80 / Pass Marks: 32\par}
	{\bfseries {BCA451CO, e-Governance \hfill}\par}
\end{framed}
% \begin{framed}
% 	\raggedright{\bfseries\Large\centering {\textnp{पूर्वाञ्चल विश्वविद्यालय}}\par {\textnp{२०१८} }\par}
% 	{ \textnp{४ वर्षे ब्याचलर अफ कम्प्युटर एप्लिकेसन (वि.सि.ए) अ‍ाठाैँ सेमेस्टर}\par}
% 	% 4 Years Bachelor of Computer Application (BCA/Eighth Semester/Final)\par
% 	{ \textnp{समयः ३ घण्टा}.  \hfill \textnp{पूर्णा‌ङ्क : ६० / उत्तीर्णाङ्क : २४}\par}
% 	{\bfseries {BCA452CO, Multimedia Application \hfill}\par}
% \end{framed}
{\noindent \it{Candidates are required to give their own answers in their own words as far as practicable.  }\par}
{\noindent \it{Figure in the margin indicate full marks.}\par}

%%%%%%%%%%%%%%%%%%%%%%%%%%%%%%%%%%%%%%%%%%%%%%%%%%%%%%%%%%%%%%%%%%%%%%%
{\fullwidth{\centering \bfseries \underline{Group A}}}
{\hspace*{-0.5cm} \bfseries Answer TWO questions.} \hfill {\( \mathbf{2\times 12 =24}\)}
%%%%%%%%%%%%%%%%%%%%%%%%%%%%%%%%%%%%%%%%%%%%%%%%%%%%%%%%%%%%%%%%%%%%%%%

\begin{questions}

	\question[2+4+6] Differentiate e-Governance and e-Government. How e-
	government helps to improve economic condition of Nepal?
	Explain Development stages of c-Government.

	\question[3+9] What is importance of data centers? Explain e-Government
	architecture and interoperability frameworks with an example in
	e-Government infrastructure development?

	\question[6+6] Define key security challenges for government system? Explain an approach to security for e-Government

	

%%%%%%%%%%%%%%%%%%%%%%%%%%%%%%%%%%%%%%%%%%%%%%%%%%%%%%%%%%%%%%%%%%%%%%%
{\fullwidth{\centering \bfseries \underline{Group B}}\par}
{\hspace*{-0.5cm}\noindent\bfseries Answer SEVEN questions.} \hfill {\( \mathbf{7 \times 8 = 56}\)}
%%%%%%%%%%%%%%%%%%%%%%%%%%%%%%%%%%%%%%%%%%%%%%%%%%%%%%%%%%%%%%%%%%%%%%%

\question[2+6] Explain the way of partnership followed by e-Government. Describe citizen-centric approach to c-Government adoption.

\question[2+6] What are the issues in e-Government readiness? Explain different steps to e-Government readiness.

\question[8] Describe emerging management issues for e-Government system. 

\question[4+4]  Describe e-Government system life-cycle. Mention design of new
e-Government system.

\question[2+6] Explain e-Government strategy and managing public data for
Government with an example.

\question[2+6] What is GIDC? Show comparative analysis of c-Government
development in India and China.

\question[8] Explain different application areas of data warehousing and data
mining in brief.

\question[4+4] Write short notes on any TWO:
\begin{parts}
	\part Security management model
	\part e-Government master plan
	\part ICT development project in Nepal
\end{parts}
\newpage
\end{questions}

\end{document}